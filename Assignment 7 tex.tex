\documentclass{beamer}

\providecommand{\pr}[1]{\ensuremath{\Pr\left(#1\right)}}
\providecommand{\qfunc}[1]{\ensuremath{Q\left(#1\right)}}
\providecommand{\sbrak}[1]{\ensuremath{{}\left[#1\right]}}
\providecommand{\lsbrak}[1]{\ensuremath{{}\left[#1\right.}}
\providecommand{\rsbrak}[1]{\ensuremath{{}\left.#1\right]}}
\providecommand{\brak}[1]{\ensuremath{\left(#1\right)}}
\providecommand{\lbrak}[1]{\ensuremath{\left(#1\right.}}
\providecommand{\rbrak}[1]{\ensuremath{\left.#1\right)}}
\providecommand{\cbrak}[1]{\ensuremath{\left\{#1\right\}}}
\providecommand{\lcbrak}[1]{\ensuremath{\left\{#1\right.}}
\providecommand{\rcbrak}[1]{\ensuremath{\left.#1\right\}}}

\let\vec\mathbf

\newcommand{\myvec}[1]{\ensuremath{\begin{pmatrix}#1\end{pmatrix}}}
\newcommand{\mydet}[1]{\ensuremath{\begin{vmatrix}#1\end{vmatrix}}}


\usetheme{CambridgeUS}


\title{AI1110\\Assignment 7 } 
\author{Burra Vishal Mathews \\ CS21BTECH11010}

\date{\today}
\logo{\large \LaTeX{}}


\begin{document}

\begin{frame}
    \titlepage
\end{frame}

\logo{}

\begin{frame}
\tableofcontents
\end{frame}

\begin{frame}{Question}
\section{Question}
\begin{block}

Classify the states of the Markov chains with the following transition probabilities :
\begin{enumerate}
    \item 
    P=\myvec{0&\frac{1}{2}&\frac{1}{2}\\
\frac{1}{2}&0&\frac{1}{2}\\
\frac{1}{2}&\frac{1}{2}&0}

\item \myvec{0&0&\frac{1}{3}&\frac{2}{3}\\
1&0&0&0\\
0&1&0&0\\
0&0&1&0}

\item \myvec{\frac{1}{2}&\frac{1}{2}&0&0&0\\
\frac{1}{2}&\frac{1}{2}&0&0&0\\
0&0&\frac{1}{3}&\frac{2}{3}&0\\
0&0&\frac{2}{3}&\frac{1}{3}&0\\
\frac{1}{3}&\frac{1}{3}&0&0&\frac{1}{3}}
\end{enumerate}
\end{block}

\end{frame}

\begin{frame}{Solution for part 1 $\And$ 2}
\section{Solution}
\begin{enumerate}

\item \begin{itemize}

    \item Period of state A = \cbrak{2,3,4 \hdots}
    \item Period of state B = \cbrak{2,3,4 \hdots}
    \item Period of state c = \cbrak{2,3,4 \hdots}
    
      Chain is irreducible and aperiodic.
    
\end{itemize}

\item \begin{itemize}
    \item Period of state A = \cbrak{3,4,6,7\hdots}
    \item Period of state B = \cbrak{3,4,6,7\hdots}
    \item Period of state C = \cbrak{3,4,6,7\hdots}
    \item Period of state D = \cbrak{3,4,6,7\hdots}
    
     Chain is irreducible and aperiodic.
\end{itemize}
\end{enumerate}
\end{frame}

\begin{frame}{Solution for part 3}
\begin{itemize}
    \item Period of state A = \cbrak{1,2,3,4\hdots}
    \item Period of state B = \cbrak{1,2,3,4\hdots}
    \item Period of state C = \cbrak{1,2,3,4\hdots}
    \item Period of state D = \cbrak{1,2,3,4\hdots}
    \item Period of state E = \cbrak{1,2,3,4\hdots}
    
    Chain has two aperiodic closed sets \cbrak{e_1,e_2} and \cbrak{e_3,e_4} and a transient state $e_5$
. 
\end{itemize}
    
\end{frame}

\end{document}